\documentclass[12pt, a4paper, twoside]{article}
\usepackage[utf8]{inputenc}

\usepackage[override]{cmtt}
\usepackage{mathpartir} % available from http://cristal.inria.fr/~remy/latex/
\usepackage{xcolor}

%%%%%%%%%%%%%%%%%%%%% BNF
\newcommand \bnfdef  {\mathrel{\color{blue}{::=}}}
\newcommand \bnfalt  {\mathrel{\color{blue}{|}}}

%%%%%%%%%%%%%%%%%%%%%
\newcommand \keyfont {\mathsf}
\newcommand \WHILE   {\keyfont{while}}
\newcommand \IF      {\keyfont{if}}
\newcommand \THEN    {\keyfont{then}}
\newcommand \ELSE    {\keyfont{else}}
\newcommand \TRUE    {\keyfont{true}}
\newcommand \FALSE   {\keyfont{false}}
\newcommand \ISZERO  {\keyfont{iszero}}
\newcommand \SKIP    {\keyfont{skip}}
\newcommand \ASS     {\texttt{:=}}
\newcommand \SEQ     {\texttt{;}}

\newcommand \while[2]   {\WHILE\ {#1}\ {#2}}
\newcommand \ifelse[3]  {\IF\ {#1}\ \THEN\ {#2}\ \ELSE\ {#3}}

\newcommand \bluearr {{\color{blue}\rightarrow}}
\newcommand \bluemarr {{\color{blue}\rightarrow^*}}
\newcommand \bluearrt[1] {{\color{blue}\xrightarrow{#1}}}
\newcommand \bluesemi {{\color{blue};}}

\newcommand \tx {\mathtt{x}}
\newcommand \ty {\mathtt{y}}
\newcommand \tz {\mathtt{z}}

\newcommand \eval[2] {{#1}\ \bluearr\ {#2}}
\newcommand \meval[2] {{#1}\ \bluemarr\ {#2}}
\newcommand \seval[4] {{#1}~|~{#2}\ \bluearr\ {#3}~|~{#4}}

\newcommand \expeval[3] {{#1}\ \bluesemi\ {#2}\
  {\color{blue}\Downarrow}\ {#3}}
\newcommand \stmteval[4] {{#1}\ \bluesemi\ {#2}\ 
  {\bluearr}\ {#3}\ \bluesemi\ {#4}}
\newcommand \stmtevalM[4] {{#1}\ \bluesemi\ {#2}\ 
  {\bluearr^{\color{blue}*}}\ {#3}\ \bluesemi\ {#4}}
\newcommand \stmtevaln[5] {{#2}\; \bluesemi \; {#3}\;\;
  {\bluearr^{\color{blue}#1}}\;\; {#4}\; \bluesemi \; {#5}}

\newcommand \noneg[1] {\mathit{noneg}(#1)}

\newcommand \denote[1] {\mathit{den}(#1)}

% for equivalence
\newcommand \expsmall[3] {{#1}{\color{blue};}\ {#2}{\color{blue}\
    \rightarrow\ }{#3}}
\newcommand \expsmallM[3]
    {{#1}{\color{blue};}\ {#2}{\color{blue}\ \rightarrow^*\ }{#3}}
\newcommand \expsmalln[4] {{#2}{\color{blue};}\ {#3}{\color{blue}\
    \rightarrow^{#1}\ }{#4}}

%%%%%%%%%%%%%%%%%%%%
\newcommand \T {\tau}
\newcommand \G {\Gamma}
\newcommand \C {\cdot}
\newcommand \A \alpha
\newcommand \B {\beta}
\newcommand \D {\Delta}
\newcommand \intty {\mathsf{int}}
\newcommand \unitty {\mathsf{unit}}
\newcommand \fnty[2] {{#1}\rightarrow {#2}}
\newcommand \lam[2] {\lambda{#1}.\ {#2}}
\newcommand \subst[3] {{#1}[{#2}/{#3}]}
\newcommand \tc[3]{{#1} \vdash {#2} : {#3}}
\newcommand \dom {\mathrm{Dom}}
\newcommand \arr {\rightarrow}

\newcommand \letin[3] {\mathsf{let}\ {#1}={#2}\ \mathsf{in}\ {#3}}
\newcommand \tru {\mathsf{true}}
\newcommand \fls {\mathsf{false}}
\newcommand \boolty {\mathsf{bool}}

\newcommand \pairty[2] {{#1}*{#2}}

%% \newcommand \inl[1] {\mathsf{inl}({#1})}
%% \newcommand \inr[1] {\mathsf{inr}({#1})}
%% \newcommand \case[4] {\mathsf{case}\ {#1}\ {#2}.{#3}\ |\ {#2}.{#4}}
\newcommand \Ac{\mathsf{A}}
\newcommand \Bc{\mathsf{B}}
\newcommand \Acons[1] {\Ac({#1})}
\newcommand \Bcons[1] {\Bc({#1})}
\newcommand \match[5] {\mathsf{match}\ {#1}\ \mathsf{with}\ \Ac {#2}.\
  {#3}\ |\ \Bc {#4}.\ {#5}}
\newcommand \sumty[2] {{#1}+{#2}}

\newcommand \fix[1] {\mathsf{fix}\ {#1}}

\newcommand \Lam[2] {\Lambda{#1}.\ {#2}}

\newcommand \primstep[2] {{#1}\stackrel{\mathrm{p}}{\rightarrow}{#2}}

\newcommand \LETCC {\mathsf{letcc}}
\newcommand \THROW {\mathsf{throw}}
\newcommand \letcc[2] {\LETCC\ {#1}.\ {#2}}
\newcommand \throw[2] {\THROW\ {#1}\ {#2}}
\newcommand \cont[1]  {\mathsf{cont}\ {#1}}

\newcommand \forallty[2] {\forall{#1}.{#2}}
\newcommand \existsty[2] {\exists{#1}.{#2}}
\newcommand \muty[2] {\mu{#1}.{#2}}

\newcommand \kc[2] {{#1}\vdash{#2}}
\newcommand \tcF[4]{{#1};{#2} \vdash {#3} : {#4}}

\newcommand \erase[1] {\mathit{erase}({#1})}


\title{Lab 1}
\author{DD2481 - Principles of Programming Languages}
\date{30 March 2021}

\begin{document}
\maketitle

\section{Goal}

The goal of this lab is the implementation of an interpreter for the
untyped lambda calculus extended with boolean and arithmetic
expressions, as seen in the lectures. We provide a code
template. Below we explain the parts of the template that you need to
be familiar with, however, we recommend that you spend some time
exploring also other parts of the template not mentioned below. (For
reference, we will refer to different files in the source
distribution; in each case we omit the common prefix
``src/main/scala/simpret/''.)

The syntax and small-step evaluation rules follow.

Syntax:

\[\begin{array}{rclr}
t &\bnfdef& ~ & \mathit{terms:} \\
~ &~      & \TRUE            & \mathit{constant~true} \\
~ &\bnfalt& \FALSE           & \mathit{constant~false} \\
~ &\bnfalt& \ifelse{t}{t}{t} & \mathit{conditional} \\
~ &\bnfalt& c                & \mathit{integer~constant} \\
~ &\bnfalt& \ISZERO~t        & \mathit{zero~test} \\
~ &\bnfalt& t + t            & \mathit{integer~addition} \\
~ &\bnfalt& x                & \mathit{variable} \\
~ &\bnfalt& \lambda x.~t     & \mathit{abstraction} \\
~ &\bnfalt& t~t              & \mathit{application} \\
\end{array}\]

\[\begin{array}{rclr}
v &\bnfdef& ~ & \mathit{values:} \\
~ &~      & \TRUE            & \mathit{true~value} \\
~ &\bnfalt& \FALSE           & \mathit{false~value} \\
~ &\bnfalt& c                & \mathit{integer~value} \\
~ &\bnfalt& \lambda x.~t     & \mathit{abstraction} \\
\end{array}\]

\[\begin{array}{rcl}
(c &{\color{blue}{\in}}& \{\dots,-2,-1,0,1,2,\dots\}) \\
(x &{\color{blue}{\in}}& \{\tx_1,\tx_2,\dots,\ty_1,\ty_2,\dots,\tz_1,\tz_2,\dots,\dots\})
\end{array}\]
\newpage
Evaluation:

\begin{mathpar}
    \infer[E-IfTrue]{}{\eval{\ifelse{\TRUE}{t_2}{t_3}}{t_2}}

    \infer[E-IfFalse]{}{\eval{\ifelse{\FALSE}{t_2}{t_3}}{t_3}} \quad

    \infer[E-If]{\eval{t_1}{t_1'}}{\eval{\ifelse{t_1}{t_2}{t_3}}{\ifelse{t_1'}{t_2}{t_3}}}

    \infer[E-IsZeroZero]{}{\eval{\ISZERO~0}{\TRUE}} \quad

    \infer[E-IsZeroNonZero]{c \neq 0}{\eval{\ISZERO~c}{\FALSE}} \quad

    \infer[E-IsZero]{\eval{t_1}{t_1'}}{\eval{\ISZERO~t_1}{\ISZERO~t_1'}} \quad

    \infer[E-Add]{}{\eval{c_1 + c_2}{c_1~\mathcal{I}(+)~c_2}} \quad

    \infer[E-AddRight]{\eval{t_2}{t_2'}}{\eval{c_1 + t_2}{c_1 + t_2'}} \quad

    \infer[E-AddLeft]{\eval{t_1}{t_1'}}{\eval{t_1 + t_2}{t_1' + t_2}} \quad

    \infer[E-App1]{\eval{t_1}{t_1'}}{\eval{t_1~t_2}{t_1'~t_2}} \quad

    \infer[E-App2]{\eval{t_2}{t_2'}}{\eval{v_1~t_2}{v_1~t_2'}} \quad

    \infer[E-AppAbs]{}{\eval{(\lambda x.~t_1)~v_2}{[x \mapsto v_2]t_1}}
  \end{mathpar}

The interpreter should be based directly on the small-step evaluation rules.

In order to be able to parse source code, we are providing you with a
code template that includes a fully-functioning and complete lexer and
parser (the implementation uses an approach called parser combinators,
provided by the Scala library at
\texttt{https://github.com/scala/scala-parser-combinators}). You do
not have to work on these parts at all. The result of parsing a
program is an abstract syntax tree (AST). The AST, which is also
provided to you, is implemented in Scala as follows (file
``parser/AST.scala''):

\newpage
\begin{verbatim}
trait AST extends Positional
case class Variable(id: String) extends AST
case class BoolLit(b: Boolean) extends AST
case class IntLit(i: Int) extends AST
case class CondExp(c: AST, e1: AST, e2: AST) extends AST
case class IsZeroExp(e: AST) extends AST
case class PlusExp(e1: AST, e2: AST) extends AST
case class LamExp(id: String, e: AST) extends AST
case class AppExp(e1: AST, e2: AST) extends AST
\end{verbatim}

Your task is to implement the \texttt{step} method of the
 \texttt{Interpreter} singleton object in file
 ``interpreter/Interpreter.scala'':

\begin{verbatim}
def step(tree: AST): Option[AST] = {
  // TODO: your implementation goes here
}
\end{verbatim}

In order to do one (small) step of computation, the idea is to match
on the AST parameter \verb|tree|:

\begin{verbatim}
tree match {
  case PlusExp(e1, e2) if !isvalue(e1) => ...
  ...
}
\end{verbatim}

For developing and testing your solution you can use the Main object
(file ``Main.scala'') and also the automated unit tests (see below).

\section{Build Tool}

The project is built using sbt, the most widely used build tool in
Scala.\footnote{See \texttt{https://www.scala-sbt.org}} Make sure to
familiarize yourself with sbt, so that you are able to compile and run
the code template. The guide ``Getting Started with sbt'' is a good
starting
point:\\ \texttt{https://www.scala-sbt.org/1.x/docs/Getting-Started.html}

You can build the project by invoking \texttt{sbt} in the root
directory of the project:

\begin{verbatim}
$ sbt
[info] Loading project definition from ...
[info] Loading settings from build.sbt ...
[info] Set current project to simpret (in build file:/...
[info] sbt server started at local:///...
sbt:simpret>
\end{verbatim}

You get to an input prompt where you can enter sbt commands, such as
\texttt{compile}, for compiling the source of the current project, and
\texttt{run}, for running the main object. Make sure that
\texttt{compile} works. (Compiling will output JVM class files into
the directory ``lab1/target/scala-2.12/classes''.) Importantly, you
can pass arguments to the \texttt{run} command. The code template
comes with a main object where you can provide the file name of a
source file that contains an expression to be interpreted.

For example, put the following expression into a file ``test.sint'':
\begin{verbatim}
if true then 3 else 4
\end{verbatim}

Then, at the sbt prompt, enter \texttt{run test.sint} and press
enter. You should see output similar to the following:

\begin{verbatim}
sbt:simpret> run test.sint
[info] Packaging /Users/.../lab1/target/scala-2.12/simpret_...
[info] Done packaging.
[info] Running simpret.Main test.sint
==========================================================
AST
==========================================================
if true then
  3
else
  4
==========================================================
Evaluation stuck!
Stuck at
==================
if true then
  3
else
  4
==================
[success] Total time: 1 s, completed Mar 27, 2019 9:35:46 PM
\end{verbatim}

It says that evaluation is stuck, because the interpreter has not been
implemented, yet. You also see that it provides a pretty-printed
rendering of the AST.

\section{Unit Testing}

The provided template is set up to enable running automated unit
tests, using a testing framework called ScalaTest. Unit tests can be
run from the sbt prompt using the ``test'' command. We provide a suite
of tests which your solution needs to pass to be accepted.

There are two kinds of unit tests. The first kind consists of a source
file located in ``src/test/sint/lab1/'', for example,
\newline``src/test/sint/lab1/case\_0\_cond\_0.sint''. This file is parsed and
evaluated using the code in
\newline``src/test/scala/simpret/MainTestLab1.scala'' and compared to an
expected outcome, in this case \verb|Right(IntLit(11))|.

The second kind of unit test consists of a pair of files located in
\newline``src/test/sint/lab1/steptests/''. For example, ``Add-0.sint''
contains the initial program, and ``Add-0.sint.step'' contains the result
of one (small) step of the evaluation. All of the (pairs of) files in
the ``steptests'' directory are run and checked automatically when you
start the unit tests in sbt.

Let's get one of the tests to pass. For this we replace the
\verb|None| in the \verb|step| method in the \verb|Interpreter| object
with the following code:

\begin{verbatim}
tree match {
  case PlusExp(IntLit(x), IntLit(y)) =>
    Some(IntLit(x + y))
  // TODO: complete implementation here
  case _ => None
}
\end{verbatim}

Good luck!

\end{document}
